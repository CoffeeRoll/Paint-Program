\mbox{\hyperlink{namespace_wintab_d_n}{Wintab\+DN}} is a wrapper of the Wintab32 A\+PI that supports writing .N\+ET compatible applications for Wacom digitizing tablets. \begin{DoxyAuthor}{Author}
Robert Cohn, Wacom Technology Corporation (\href{mailto:rcohn@wacom.com}{\tt rcohn@wacom.\+com})
\end{DoxyAuthor}
\begin{DoxyParagraph}{Revision History}
\tabulinesep=1mm
\begin{longtabu} spread 0pt [c]{*{4}{|X[-1]}|}
\hline
\rowcolor{\tableheadbgcolor}\textbf{ Revision Date}&\textbf{ Revisor\textquotesingle{}s Name (Email)}&\textbf{ Change Description}&\textbf{ Version  }\\\cline{1-4}
\endfirsthead
\hline
\endfoot
\hline
\rowcolor{\tableheadbgcolor}\textbf{ Revision Date}&\textbf{ Revisor\textquotesingle{}s Name (Email)}&\textbf{ Change Description}&\textbf{ Version  }\\\cline{1-4}
\endhead
11/13/2010&Robert Cohn (\href{mailto:rcohn@wacom.com}{\tt rcohn@wacom.\+com})&Initial Version&1.\+0  \\\cline{1-4}
03/15/2013&Robert Cohn (\href{mailto:rcohn@wacom.com}{\tt rcohn@wacom.\+com})&Added Wintab Extensions Support&1.\+1  \\\cline{1-4}
08/13/2013&Robert Cohn (\href{mailto:rcohn@wacom.com}{\tt rcohn@wacom.\+com})&Now requires .N\+ET 4; fixed Wt\+Packets\+Get testing; marshalling improvements (from \href{mailto:brett@brett.net.au}{\tt brett@brett.\+net.\+au})&1.\+2  \\\cline{1-4}
\end{longtabu}

\end{DoxyParagraph}
\hypertarget{index_intro_sec}{}\section{Introduction}\label{index_intro_sec}
The Wintab32 A\+PI (originally developed by L\+C\+S/\+Telegraphics in the early 1990s) was created to provide a standardized programming interface to digitizing tablets, and was early adopted by Wacom Technology Corporation to support writing Windows operating system native C++ applications for its pen digitizing tablets. A complete description of the Wintab32 A\+PI can be found in the \href{Wintab_v140.htm}{\tt Wintab 1.\+4 specification}.

The \mbox{\hyperlink{namespace_wintab_d_n}{Wintab\+DN}} A\+PI was created to aid the development of managed code applications for Wacom\textquotesingle{}s digital tablets. This new A\+PI is .N\+ET 2 compatible and will support the writing of applications in any .N\+ET supported language (such as C\# or V\+B.\+N\+ET).

With \mbox{\hyperlink{namespace_wintab_d_n}{Wintab\+DN}}, an application developer can, for example, easily write a .N\+ET application to set up and capture pen data indicating X/Y location and pressure. Other applications can be written to monitor pen tilt or rotation. The A\+PI can be incorporated into many software applications where precise pen location data would be useful (such as M\+A\+T\+L\+AB, or C\+AD applications).

\mbox{\hyperlink{namespace_wintab_d_n}{Wintab\+DN}} is a work in progress. Version 1.\+0 only wraps a subset of the extensive Wintab32 native implementation, and it is hoped that a growing community of developers will use \mbox{\hyperlink{namespace_wintab_d_n}{Wintab\+DN}} and contribute to its maintenance and extension.\hypertarget{index_contact_sec}{}\section{Contact Info}\label{index_contact_sec}
If you have questions about using \mbox{\hyperlink{namespace_wintab_d_n}{Wintab\+DN}}, you can send email to this address\+: \href{mailto:DeveloperEmailGroup@wacom.com}{\tt Developer\+Email\+Group@wacom.\+com}

Also, visit \href{http://wacomeng.com/windows/index.html}{\tt Wacom Software Developer Support} for general Wintab32 tablet programming support. 